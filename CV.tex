%%%%%%%%%%%%%%%%%
% This is an sample CV template created using altacv.cls
% (v1.7.2, 28 August 2024) written by LianTze Lim (liantze@gmail.com). Compiles with pdfLaTeX, XeLaTeX and LuaLaTeX.
%
%% It may be distributed and/or modified under the
%% conditions of the LaTeX Project Public License, either version 1.3
%% of this license or (at your option) any later version.
%% The latest version of this license is in
%%    http://www.latex-project.org/lppl.txt
%% and version 1.3 or later is part of all distributions of LaTeX
%% version 2003/12/01 or later.
%%%%%%%%%%%%%%%%

%% Use the "normalphoto" option if you want a normal photo instead of cropped to a circle
% \documentclass[10pt,a4paper,normalphoto]{altacv}

\documentclass[10pt,a4paper,ragged2e,withhyper]{altacv/altacv}
%% AltaCV uses the fontawesome5 and simpleicons packages.
%% See http://texdoc.net/pkg/fontawesome5 and http://texdoc.net/pkg/simpleicons for full list of symbols.

% Change the page layout if you need to
\geometry{left=1.25cm,right=1.25cm,top=1cm,bottom=1.25cm,columnsep=0.8cm}

% The paracol package lets you typeset columns of text in parallel
\usepackage{paracol}

% Change the font if you want to, depending on whether
% you're using pdflatex or xelatex/lualatex
% WHEN COMPILING WITH XELATEX PLEASE USE
% xelatex -shell-escape -output-driver="xdvipdfmx -z 0" sample.tex
\iftutex
  % If using xelatex or lualatex:
  \setmainfont{Roboto Slab}
  \setsansfont{Lato}
  \renewcommand{\familydefault}{\sfdefault}
\else
  % If using pdflatex:
  \usepackage[rm]{roboto}
  \usepackage[defaultsans]{lato}
  % \usepackage{sourcesanspro}
  \renewcommand{\familydefault}{\sfdefault}
\fi

\definecolor{SlateGrey}{HTML}{2E2E2E}
\definecolor{LightGrey}{HTML}{666666}
\definecolor{DarkPastelBlue}{HTML}{0e384a}
\definecolor{PastelBlue}{HTML}{166c92}
\definecolor{GoldenEarth}{HTML}{E7D192}
\colorlet{name}{black}
\colorlet{tagline}{PastelBlue}
\colorlet{heading}{DarkPastelBlue}
\colorlet{headingrule}{PastelBlue}
\colorlet{subheading}{PastelBlue}
\colorlet{accent}{PastelBlue}
\colorlet{emphasis}{SlateGrey}
\colorlet{body}{LightGrey}

% Change some fonts, if necessary
\renewcommand{\namefont}{\Huge\rmfamily\bfseries}
\renewcommand{\personalinfofont}{\footnotesize}
\renewcommand{\cvsectionfont}{\LARGE\rmfamily\bfseries}
\renewcommand{\cvsubsectionfont}{\large\bfseries}


% Change the bullets for itemize and rating marker
% for \cvskill if you want to
\renewcommand{\cvItemMarker}{{\small\textbullet}}
\renewcommand{\cvRatingMarker}{\faCircle}
% ...and the markers for the date/location for \cvevent
% \renewcommand{\cvDateMarker}{\faCalendar*[regular]}
% \renewcommand{\cvLocationMarker}{\faMapMarker*}


% If your CV/résumé is in a language other than English,
% then you probably want to change these so that when you
% copy-paste from the PDF or run pdftotext, the location
% and date marker icons for \cvevent will paste as correct
% translations. For example Spanish:
% \renewcommand{\locationname}{Ubicación}
% \renewcommand{\datename}{Fecha}


%% Use (and optionally edit if necessary) this .tex if you
%% want to use an author-year reference style like APA(6)
%% for your publication list
% \input{pubs-authoryear.cfg}

%% Use (and optionally edit if necessary) this .tex if you
%% want an originally numerical reference style like IEEE
%% for your publication list
\input{altacv/pubs-num.cfg}

%% sample.bib contains your publications
\addbibresource{sample.bib}

\usepackage{tikz}
\usepackage{textpos} % Pozwala precyzyjnie ustawić tekst na stronie


\begin{document}




\name{Dawid Zubrycki}
\tagline{Developer and Proud Geek}
%% You can add multiple photos on the left or right

\photoL{2.5cm}{picture}
% \photoL{2.5cm}{Yacht_High,Suitcase_High}

\personalinfo{%
  % Not all of these are required!
  \email{Dawid_00@outlook.com}
  \github{dzubrycki}
  \phone{+48 600 012 782}
  % \mailaddress{}
  \location{Stefana Okrzei 13/13, 81-225 Gdynia}
  % \homepage{www.homepage.com}
  % \twitter{@twitterhandle}
  % \xtwitter{@x-handle}
  % \linkedin{your_id}
  % \orcid{0000-0000-0000-0000}
  %% You can add your own arbitrary detail with
  %% \printinfo{symbol}{detail}[optional hyperlink prefix]
  % \printinfo{\faPaw}{Hey ho!}[https://example.com/]

  %% Or you can declare your own field with
  %% \NewInfoFiled{fieldname}{symbol}[optional hyperlink prefix] and use it:
  % \NewInfoField{gitlab}{\faGitlab}[https://gitlab.com/]
  % \gitlab{your_id}
  %%
  %% For services and platforms like Mastodon where there isn't a
  %% straightforward relation between the user ID/nickname and the hyperlink,
  %% you can use \printinfo directly e.g.
  % \printinfo{\faMastodon}{@username@instace}[https://instance.url/@username]
  %% But if you absolutely want to create new dedicated info fields for
  %% such platforms, then use \NewInfoField* with a star:
  % \NewInfoField*{mastodon}{\faMastodon}
  %% then you can use \mastodon, with TWO arguments where the 2nd argument is
  %% the full hyperlink.
  % \mastodon{@username@instance}{https://instance.url/@username}
}

\makecvheader
%% Depending on your tastes, you may want to make fonts of itemize environments slightly smaller
% \AtBeginEnvironment{itemize}{\small}

%% Set the left/right column width ratio to 6:4.
\columnratio{0.6}

% Start a 2-column paracol. Both the left and right columns will automatically
% break across pages if things get too long.
\begin{paracol}{2}

\cvsection{Core competencies}

\begin{itemize}
\item Proficient in Linux command-line environment
\item Work independently, identify needs, optimize processes
\item Goal-driven with focus on achieving results
\item Understanding of internal process from 6 years as a CW
\item Maintained strong working relationships with DevOps teams
\item Experienced in writing clear and concise documentation
\item Ability to identify and resolve issues in Jenkins pipelines
\item Strong sense of responsibility and ownership in assigned tasks
\item Effective communication within my team and across departments
\item Teaching others and sharing acquired knowledge
\item Experience and daily working with AI frameworks (pyTorch)
\item Familiar with UI/UX development
\end{itemize}

\cvsection{Experience}

\large\textbf{ManpowerGroup Sp. z o.o.} \normalsize

\medskip

\textbf{\textcolor{accent}{GDN QA - CW team leader}} \hfill \cvDateMarker {2022 -- current}

\smallskip

\begin{itemize}
\item Automating repetitive tasks with Python
\item Verifying results from nightly tests (functional and performance)
\item Advanced bisection, reporting unwanted behavior
\item Contingent Worker team leader, recruiting new team members
\end{itemize}

\medskip

\textbf{\textcolor{accent}{Remote servers’ maintenance}} \hfill \cvDateMarker {2021 -- 2022}
\smallskip

\begin{itemize}
\item Automated remote maintenance of 40 Linux Gaudi machines
\item Reporting servers/Gaudi issues and fixing when possible
\item Helping developers with Gaudi driver on the machines
\end{itemize}

\medskip

\textbf{\textcolor{accent}{Lab technician}} \hfill \cvDateMarker {2019 -- 2021}
\smallskip

\begin{itemize}
\item Maintenance of 2 labs (more than 100 baremetal servers)
\item Implementing and configuring physical and virtual systems
\item Debugging malfunction hardware issues
\end{itemize}

% \cvsection{Projects}

% \cvevent{Project 1}{Funding agency/institution}{}{}
% \begin{itemize}
% \item Details
% \end{itemize}

% \divider

% \cvevent{Project 2}{Funding agency/institution}{Project duration}{}
% A short abstract would also work.

\medskip

\cvsection{Education}

\large\textbf{Engineer in Computer Science} \normalsize \hfill \cvLocationMarker {Gdynia}\\
\textbf{\textcolor{accent}{WSB Merito University}} \hfill \cvDateMarker {2019 -- 2024}\\
\smallskip
Thesis title: “Design and Implementation of an Energy-Efficient\\Homelab Infrastructure with Advanced User Applications”

\divider

\large\textbf{Mechatronics Technician} \normalsize \hfill \cvLocationMarker {Giżycko}\\
\textbf{\textcolor{accent}{KEN School of Electronics and Computer Science}} \hfill \cvDateMarker {2015 -- 2019}\\
\smallskip


\cvsection{Personal Projects}

\begin{itemize}
\item Proud Homelab owner and member of self-hosted community
\item Managing 20 services in my own server infrastructure
\end{itemize}



% \cvsection{A Day of My Life}

% Adapted from @Jake's answer from http://tex.stackexchange.com/a/82729/226
% \wheelchart{outer radius}{inner radius}{
% comma-separated list of value/text width/color/detail}
% \wheelchart{1.5cm}{0.5cm}{%
%   6/8em/accent!30/{Sleep,\\beautiful sleep},
%   3/8em/accent!40/3D printing and self-hosting hobbyist by night,
%   8/8em/accent!60/Daytime job,
%   2/10em/accent/Gym and relaxation,
%   5/6em/accent!20/Family time and self-development
% }
% use ONLY \newpage if you want to force a page break for
% ONLY the current column
% \newpage

% \cvsection{Publications}

% %% Specify your last name(s) and first name(s) as given in the .bib to automatically bold your own name in the publications list.
% %% One caveat: You need to write \bibnamedelima where there's a space in your name for this to work properly; or write \bibnamedelimi if you use initials in the .bib
% %% You can specify multiple names, especially if you have changed your name or if you need to highlight multiple authors.
% \mynames{Lim/Lian\bibnamedelima Tze,
%   Wong/Lian\bibnamedelima Tze,
%   Lim/Tracy,
%   Lim/L.\bibnamedelimi T.}
% %% MAKE SURE THERE IS NO SPACE AFTER THE FINAL NAME IN YOUR \mynames LIST

% \nocite{*}

% \printbibliography[heading=pubtype,title={\printinfo{\faBook}{Books}},type=book]

% \divider

% \printbibliography[heading=pubtype,title={\printinfo{\faFile*[regular]}{Journal Articles}},type=article]

% \divider

% \printbibliography[heading=pubtype,title={\printinfo{\faUsers}{Conference Proceedings}},type=inproceedings]

% %% Switch to the right column. This will now automatically move to the second
% %% page if the content is too long.
\switchcolumn

\cvsection{Most Proud of}

\cvachievement{\faTrophy}{Automated maintenance with Ansible}{Ensuring servers remain consistently aligned, updated and clean with minimal effort. Reducing manual work and enabling me to support the GDN QA team.}

\divider

\cvachievement{\faTrophy}{Development Python tool for QA team}{Writing code in Python that collects data from Jenkins and creates Excel reports, allowing better score validation and ad-hoc data for meetings.}

\divider

\cvachievement{\faHeartbeat}{Fruitful teamwork over the years}{Clear and friendly communication with various teams, resulting in continuous improvements and the exchange of valuable experiences for both sides.}


\cvsection{Strengths}
% Don't overuse these \cvtag boxes — they're just eye-candies and not essential. If something doesn't fit on a single line, it probably works better as part of an itemized list (probably inlined itemized list), or just as a comma-separated list of strengths.
\cvtag{Team-player}
\cvtag{Critical thinking}
\cvtag{Self-discipline}
\cvtag{Responsible \& Scrupulous}
\cvtag{Positive \& Composed}

\cvsection{Knowledge}
\textbf{Core:} \normalsize\smallskip\\
\cvtag{Jenkins}
\cvtag{Jira}
\cvtag{Kibana}
\cvtag{Python}
\cvtag{Grafana}
\cvtag{Github Actions}
\cvtag{KubeVirt}
\cvtag{Linux}
\medskip

\textbf{Linux OS management:} \normalsize\smallskip\\
\cvtag{Python}
\cvtag{Ansible}
\cvtag{ssh}
\cvtag{cli}
\cvtag{bash}
\medskip

\textbf{Software Engineering Practices:} \normalsize\smallskip\\
\cvtag{git}
\cvtag{IaC}
\cvtag{TDD}
\cvtag{SDLC}
\cvtag{Agile}
\medskip

\textbf{Virtualization/contenerization:} \normalsize\smallskip\\
\cvtag{Docker}
\cvtag{ContainerD}
\cvtag{LXC}
\cvtag{KubeVirt}
\cvtag{QEMU}
\cvtag{K8S}
\cvtag{Proxmox}
\cvtag{Vagrant}


\cvsection{Languages}

\cvskill{Polish (Native)}{5}
\cvskill{English (C1)}{4}

% \divider
% \cvskill{German}{3.5} %% Supports X.5 values.

%% Yeah I didn't spend too much time making all the
%% spacing consistent... sorry. Use \smallskip, \medskip,
%% \bigskip, \vspace etc to make adjustments.
\medskip

\cvsection{Personal data} 
I hereby give consent for my personal data included in my application to be processed for the purposes of the recruitment process.

% \divider

% \cvevent{Mechatronics Technician}{KEN School of Electronics and Computer Science, Giżycko}{2015 -- 2019}{}

% \divider

% \cvevent{B.Sc.\ in Your Discipline}{Stanford University}{Sept 1998 -- June 2001}{}

% % \divider

% \cvsection{Referees}

% % \cvref{name}{email}{mailing address}
% \cvref{Prof.\ Alpha Beta}{Institute}{a.beta@university.edu}
% {Address Line 1\\Address line 2}

% \divider

% \cvref{Prof.\ Gamma Delta}{Institute}{g.delta@university.edu}
% {Address Line 1\\Address line 2}



% \end{paracol}

% \end{document}

\end{paracol}

\begin{textblock*}{\textwidth}(0cm,1cm)
  \centering
  {\scriptsize This CV was made in pure \LaTeX}
\end{textblock*}

\end{document}
